\documentclass[12pt]{article}
%%%%%%%%%%%%%%%%%%%%%%%%%%%%%%%%%%%%%%%%%%%%%%%%%%%%%%%%%%%%%%%%%%%%%%%%%%%%%%%%%
% Packages
\usepackage{epsfig} \usepackage{graphicx} \usepackage{amsmath}
\usepackage{fancyhdr} \usepackage{amsfonts} \usepackage{fullpage}
\usepackage{slashed} \usepackage{feynmp} \usepackage{slashed}
\usepackage{amssymb}


% Shortcuts
\newcommand{\da}{\dagger}
\newcommand{\del}{\partial}
\newcommand{\dm}{\del_\mu}
\newcommand{\mat}[1]{\begin{pmatrix} #1 \end{pmatrix}}
\newcommand{\fs}[1]{\slashed{#1}}
\newcommand*\dif{\mathop{}\!\mathrm{d}}
\DeclareMathOperator{\Tr}{Tr}
\DeclareMathOperator{\Li_2}{Li_2}
\providecommand*{\unit}[1]{\ensuremath{\mathrm{\,#1}}}
% The imaginary unit
\providecommand*{\iu}%
{\ensuremath{\mathrm{i}}}
\newcommand{\ve}{\varepsilon}
\DeclareMathOperator{\sign}{sign}
\newcommand{\ep}{\epsilon}


\begin{document}

\noindent 1. [d] \\
2. [a] \\
3. [d] \\
4. [d] \\
5. [c] [use ${(1-1/N)}^{N}\sim{}e$]\\
6. [e] Get one point for matching to possible target function on any
component of remaining vector x. Then given a hypothesis, there will
always be $2^{3-1}=4$ matches on the first component, 4 on the second,
4 on the third for a score of 12. \\
7.


\end{document}
